% ==========================================================
% DHRUV CHAUDHARY — ENGINEERING PORTFOLIO
% Professional template for Overleaf
% ==========================================================
\documentclass[11pt]{article}

% ---------- Core packages ----------
\usepackage[letterpaper,margin=0.75in]{geometry}
\usepackage{graphicx}
\graphicspath{{img/}}
\usepackage[dvipsnames]{xcolor}
\usepackage{hyperref}
\usepackage{titlesec}
\usepackage{enumitem}
\usepackage{tabularx}
\usepackage{array}
\usepackage{microtype}
\usepackage[T1]{fontenc}
\usepackage[utf8]{inputenc}
\usepackage[most]{tcolorbox}
\usepackage{multicol}
\usepackage{tikz}
\usepackage{wrapfig}

% ---------- Colors (Professional Engineering Palette) ----------
\definecolor{primarydark}{RGB}{13,27,42}      % Deep navy
\definecolor{primarygold}{RGB}{191,140,0}     % Vibrant Purdue gold
\definecolor{accentgold}{RGB}{207,185,145}    % Soft gold for accents
\definecolor{secondarygray}{RGB}{60,64,67}    % Darker gray for contrast
\definecolor{lightgray}{RGB}{248,249,250}     % Subtle background
\definecolor{dividercolor}{RGB}{218,220,224}  % Clean dividers
\definecolor{linkblue}{RGB}{26,115,232}       % Modern link blue
\definecolor{successgreen}{RGB}{15,157,88}    % For highlights

% ---------- Hyperref ----------
\hypersetup{
  colorlinks=true,
  urlcolor=linkblue,
  linkcolor=primarydark,
  citecolor=primarydark,
  pdfborder={0 0 0}
}
\urlstyle{same}

% ---------- Global settings ----------
\setlength{\parindent}{0pt}
\setlength{\parskip}{6pt}
\emergencystretch=3em
\setlist{nosep, leftmargin=1.2em}

% ---------- Section formatting ----------
\titleformat{\section}
  {\Large\bfseries\color{primarydark}\sffamily}
  {}{0pt}{}[\vspace{2pt}{\color{primarygold}\titlerule[2.5pt]}]
\titlespacing*{\section}{0pt}{0pt}{6pt}

\titleformat{\subsection}
  {\large\bfseries\color{secondarygray}}
  {}{0pt}{}
\titlespacing*{\subsection}{0pt}{10pt}{6pt}

% ---------- Custom commands ----------
\newcolumntype{Y}{>{\centering\arraybackslash}X}

% green item that survives page breaks (text) — pair with green bullet label
\newcommand{\gitem}{\item \leavevmode\color{successgreen}}

\newcommand{\ImgCapCell}[2]{%
  \begin{minipage}[t]{\linewidth}
    \centering
    \ProjectImage{#1}\\[4pt]
    {\footnotesize\color{secondarygray}\textit{#2}}
  \end{minipage}%
}

\newcommand{\divider}{%
  \vspace{6pt}{\color{dividercolor}\hrule height 0.5pt}\vspace{6pt}%
}

\newcommand{\Tag}[1]{%
  \tcbox[on line,
         boxsep=2pt,
         left=3pt,right=3pt,top=2pt,bottom=2pt,
         boxrule=0pt,
         colback=primarygold!15,
         colframe=primarygold!15,
         arc=3pt]{{\small\sffamily\bfseries\color{primarydark}#1}}%
}

\newcommand{\LinkTag}[2]{%
  \href{#1}{\Tag{#2}}%
}

% Safe image placeholder
\newcommand{\ProjectImage}[1]{%
  \IfFileExists{img/#1}{%
    \includegraphics[width=\linewidth,keepaspectratio]{img/#1}%
  }{%
    \fcolorbox{dividercolor}{lightgray}{%
      \parbox[c][4cm]{\dimexpr\linewidth-2\fboxsep-2\fboxrule\relax}{%
        \centering\small\color{secondarygray}%
        Image: \texttt{#1}\\[4pt]
        (place in \texttt{img/} folder)
      }%
    }%
  }%
}

% Large image for additional projects (bigger placeholder)
\newcommand{\ProjectImageLarge}[1]{%
  \IfFileExists{img/#1}{%
    \includegraphics[width=\linewidth,height=8cm,keepaspectratio]{img/#1}%
  }{%
    \fcolorbox{dividercolor}{lightgray}{%
      \parbox[c][6cm]{\dimexpr\linewidth-2\fboxsep-2\fboxrule\relax}{%
        \centering\small\color{secondarygray}%
        Image: \texttt{#1}\\[4pt]
        (place in \texttt{img/} folder)
      }%
    }%
  }%
}

% Image with caption
\newcommand{\CaptionedImage}[2]{%
  \begin{center}
    \ProjectImage{#1}\\[4pt]
    {\small\color{secondarygray}\textit{#2}}
  \end{center}
}

% Large image with caption for additional projects
\newcommand{\CaptionedImageLarge}[2]{%
  \begin{center}
    \ProjectImageLarge{#1}\\[4pt]
    {\small\color{secondarygray}\textit{#2}}
  \end{center}
}

% Image grid for featured projects (3 images in a row)
\newcommand{\ImageRowThree}[3]{%
  \vspace{4pt}
  \begin{tabularx}{\textwidth}{@{}XXX@{}}
    \centering\ProjectImage{#1} & 
    \centering\ProjectImage{#2} & 
    \centering\ProjectImage{#3}
  \end{tabularx}
  \vspace{8pt}
}

% Image grid with captions (3 images in a row)
\newcommand{\ImageRowThreeCaptioned}[6]{%
  \vspace{4pt}
  \begin{tabularx}{\textwidth}{@{}YYY@{}}
    \ImgCapCell{#1}{#4} & \ImgCapCell{#2}{#5} & \ImgCapCell{#3}{#6}
  \end{tabularx}
  \vspace{8pt}
}

% Image grid (2 images in a row)
\newcommand{\ImageRowTwo}[2]{%
  \vspace{4pt}
  \begin{tabularx}{\textwidth}{@{}XX@{}}
    \centering\ProjectImage{#1} & 
    \centering\ProjectImage{#2}
  \end{tabularx}
  \vspace{8pt}
}

% Image grid with captions (2 images in a row)
\newcommand{\ImageRowTwoCaptioned}[4]{%
  \vspace{4pt}
  \begin{tabularx}{\textwidth}{@{}YY@{}}
    \ImgCapCell{#1}{#3} & \ImgCapCell{#2}{#4}
  \end{tabularx}
  \vspace{8pt}
}

% Single large image
\newcommand{\SingleImage}[1]{%
  \vspace{4pt}
  \begin{center}
    \ProjectImage{#1}
  \end{center}
  \vspace{8pt}
}

% Single image with caption
\newcommand{\SingleImageCaptioned}[2]{%
  \vspace{4pt}
  \CaptionedImage{#1}{#2}
  \vspace{8pt}
}

% ---------- Header ----------
\newcommand{\PortfolioHeader}[6]{%
  \begin{center}
    {\Huge\bfseries\color{primarydark} #1}\\[6pt]
    {\large\color{secondarygray} #2}\\[10pt]
    \begin{tabular}{c}
      \color{secondarygray}
      \texttt{#3} \quad $$ \quad #4 \quad $$ \quad U.S.\ Citizen \\[3pt]
      \href{#5}{\color{linkblue}\textbf{GitHub}} \quad $\textbar$ \quad 
      \href{#6}{\color{linkblue}\textbf{LinkedIn}}
    \end{tabular}
  \end{center}
  \vspace{-6pt}
  {\color{primarygold}\hrule height 2.5pt}
  \vspace{12pt}
}

% ---------- Featured Project Environment (Full 1-2 pages) ----------
\newenvironment{FeaturedProject}[6]{%
  \begin{tcolorbox}[
    enhanced,
    colback=primarydark!3,
    colframe=primarygold,
    boxrule=2pt,
    arc=4pt,
    left=14pt,right=14pt,top=14pt,bottom=14pt,
    shadow={3pt}{-3pt}{0pt}{black!8},
    borderline west={4pt}{0pt}{primarygold},
    breakable
  ]
    {\LARGE\bfseries\color{primarydark} #1}\\[4pt]
    {\large\color{primarygold}\textbf{\textit{#2}}}\\[12pt]
    
    {\small\bfseries\color{secondarygray}TECH STACK:}\\
    {\color{primarydark}#3}\\[8pt]
    
    {\small\bfseries\color{secondarygray}CHALLENGE:}\\
    {\color{primarydark}#4}\\[8pt]
    
    {\small\bfseries\color{secondarygray}APPROACH:}\\
    {\color{primarydark}#5}\\[8pt]
    
    {\small\bfseries\color{secondarygray}IMPACT:}\\[-2pt]
    \begin{itemize}[leftmargin=*, itemsep=0pt, topsep=2pt, label=\textcolor{successgreen}{\textbullet}]
    #6
    \end{itemize}
  \end{tcolorbox}
  \vspace{10pt}
}{%
  \vspace{10pt}
}

% ---------- Project Card (Compact for Additional Projects) ----------
% Standard version with single image
\newcommand{\ProjectCard}[8]{%
  \begin{tcolorbox}[
    enhanced,
    colback=white,
    colframe=dividercolor,
    boxrule=1pt,
    arc=4pt,
    left=12pt,right=12pt,top=12pt,bottom=12pt,
    breakable,
    shadow={2pt}{-2pt}{0pt}{black!5}
  ]
    % Title and context
    {\Large\bfseries\color{primarydark} #1}\\[3pt]
    {\footnotesize\color{primarygold}\textbf{\textit{#2}}}\\[10pt]
    
    % Technologies section
    {\scriptsize\textbf{\color{secondarygray}TECHNOLOGIES}}\\[-2pt]
    {\color{primarygold}\rule{0.8cm}{1.5pt}}\\[3pt]
    {\small #3}\\[8pt]
    
    % Challenge section
    {\scriptsize\textbf{\color{secondarygray}CHALLENGE}}\\[-2pt]
    {\color{primarygold}\rule{0.7cm}{1.5pt}}\\[3pt]
    {\small #4}\\[8pt]
    
    % Solution & Impact section
    {\scriptsize\textbf{\color{secondarygray}SOLUTION \& IMPACT}}\\[-2pt]
    {\color{primarygold}\rule{1.2cm}{1.5pt}}\\[3pt]
    {\small #5}\\[10pt]
    
    % Links
    #7\\[12pt]
    
    % Image separator line
    {\color{dividercolor}\hrule height 0.5pt}\\[10pt]
    
    % Image (full width, large) with caption
    \CaptionedImageLarge{#6}{#8}
  \end{tcolorbox}
  \vspace{10pt}
}

% Multi-image version for Additional Projects
% (Links appear INSIDE the box, directly under the subheading)
\newenvironment{ProjectCardMultiImage}[6]{%
  \def\CardLinksTmp{#6}%
  \begin{tcolorbox}[
    enhanced,
    colback=white,
    colframe=dividercolor,
    boxrule=1pt,
    arc=4pt,
    left=12pt,right=12pt,top=12pt,bottom=12pt,
    breakable,
    shadow={2pt}{-2pt}{0pt}{black!5}
  ]
    % Title and context
    {\Large\bfseries\color{primarydark} #1}\\[3pt]
    {\footnotesize\color{primarygold}\textbf{\textit{#2}}}\\[6pt]
    \noindent\CardLinksTmp\\[10pt]
    
    % Technologies section
    {\scriptsize\textbf{\color{secondarygray}TECHNOLOGIES}}\\[-2pt]
    {\color{primarygold}\rule{0.8cm}{1.5pt}}\\[3pt]
    {\small #3}\\[8pt]
    
    % Challenge section
    {\scriptsize\textbf{\color{secondarygray}CHALLENGE}}\\[-2pt]
    {\color{primarygold}\rule{0.7cm}{1.5pt}}\\[3pt]
    {\small #4}\\[8pt]
    
    % Solution & Impact section
    {\scriptsize\textbf{\color{secondarygray}SOLUTION \& IMPACT}}\\[-2pt]
    {\color{primarygold}\rule{1.2cm}{1.5pt}}\\[3pt]
    {\small #5}\\[10pt]
    
    % Content area for multiple images (user adds them)
}{%
  \end{tcolorbox}
  \vspace{10pt}
}

% ---------- Skills Section ----------
\newcommand{\SkillCategory}[2]{%
  \textbf{\color{primarydark}#1} \quad {\small\color{secondarygray}#2}
}

% ==========================================================
% DOCUMENT CONTENT
% ==========================================================
\begin{document}

\PortfolioHeader
  {Dhruv Chaudhary}
  {Software-Focused Computer Engineer — Embedded Systems, RF \& Cloud}
  {dhruv.chaudhary@hotmail.com}
  {425-591-6187}
  {https://github.com/Duhuhruv}
  {https://linkedin.com/in/DhruvChaudhary1215}

\vspace{-8pt}

% ---- About Section ----
\section*{About}
Recent Purdue Computer Engineering graduate with a proven track record in both software and hardware domains. Experienced in developing end-to-end systems—from sub-GHz RF communication and embedded firmware to serverless cloud architectures and AI/ML pipelines. Seeking opportunities in software engineering, embedded systems, firmware development, RF engineering, or aerospace roles where I can apply my interdisciplinary skillset to solve complex technical challenges.

\textbf{Technical Strengths:} Full-stack development, distributed systems, embedded C/C++ programming, wireless protocols, cloud infrastructure (AWS), reinforcement learning, and cross-functional collaboration.

% ============================================================
% FEATURED PROJECT 1: DRONE TELEMETRY (1-2 PAGES)
% ============================================================
\section*{Featured Projects}

\begin{FeaturedProject}
  {Sub-GHz RF Telemetry System for Search \& Rescue Drone}
  {Senior Design Project — Communications Subteam Lead}
  {C, Python, TI CC1312R SimpleLink, UART, Custom Protocol Design, JPEG Compression, Luckfox Pico Ultra}
  {Design and implement a robust, long-range wireless communication system capable of transmitting real-time compressed images and alert data from an autonomous search-and-rescue drone with onboard object detection to a ground station over distances exceeding 3 km.}
  {Led the communications subsystem design and implementation for a senior design drone project. Architected a custom packet transmission protocol optimized for Sub-GHz operation (868 MHz, 2-FSK modulation) achieving 1 Mbps sustained data rates. Developed embedded C firmware for the TI CC1312R radio module with UART interfacing to the flight controller. Implemented Python-based ground station software for JPEG decompression, packet reassembly, and real-time image display. Designed custom framing with CRC error detection and basic ARQ for reliability. Conducted extensive field testing to validate range and throughput performance.}
  {%
    \gitem 3+ km operational range validated in line-of-sight field tests
    \gitem 1 Mbps sustained data rate for image transmission
    \gitem Custom packet protocol with CRC/ARQ for reliability
    \gitem Successfully integrated with YOLOv5 object detection pipeline
  }
  
  \vspace{6pt}
  \LinkTag{https://github.com/Duhuhruv/Sub-GHz_Radio_Drone_Project}{GitHub Repository}
  \LinkTag{https://drive.google.com/file/d/1hLyYco8pAcUMKnUi00cZ6kdRbGmCKY5S/view?usp=sharing}{Design Document}
\end{FeaturedProject}

\subsection*{System Architecture \& Implementation}

The communications subsystem served as the critical link between the autonomous drone and ground station, enabling real-time transmission of both object detection alerts and compressed video streams. The system architecture consisted of three primary components:

\textbf{1. Embedded Transmitter (Drone-Side):}
\begin{itemize}
  \item TI CC1312R SimpleLink wireless MCU configured for 868 MHz Sub-GHz operation
  \item Custom C firmware implementing packet framing, CRC calculation, and UART buffering
  \item Integration with Luckfox Pico Ultra for JPEG compression and frame capture
  \item Real-time prioritization: alert packets transmitted with higher priority than image data
\end{itemize}

\textbf{2. Custom RF Protocol:}
\begin{itemize}
  \item Packet structure: Header (16 bytes) + Payload (variable) + CRC16
  \item Support for both alert messages (low-latency, <50ms) and image chunks (throughput-optimized)
  \item Basic ARQ implementation with selective retransmission for corrupted packets
  \item Achieved 99.2\% packet delivery rate in field conditions
\end{itemize}

\textbf{3. Ground Station Software (Python):}
\begin{itemize}
  \item Real-time packet reception and reassembly
  \item JPEG decompression and image display pipeline
  \item Logging and diagnostics for performance analysis
\end{itemize}

% Image placeholders - replace with your actual images
% Example with captions:
\CaptionedImage{subsystem_block_radio.jpg}{Radio Subsystem block diagram}
\CaptionedImage{ui.png}{Real time Image transfer and UI example from Purdue SPARK Demo}

% Or without captions:
% \ImageRowThree{drone_system_diagram.jpg}{cc1312r_hardware.jpg}{field_test_setup.jpg}

\subsection*{Technical Challenges \& Solutions}

\textbf{Challenge 1: Balancing Latency vs. Throughput}

Object detection alerts required low-latency transmission (<100ms), while image data needed high throughput (1 Mbps). Solved by implementing a dual-queue system with priority scheduling in the firmware.

\textbf{Challenge 2: Packet Loss in Long-Range Scenarios}

Initial testing at 2+ km showed 15-20\% packet loss due to signal fading. Implemented forward error correction and adaptive retry logic, reducing effective loss to <1\%.

\textbf{Challenge 3: Power Constraints}

CC1312R power consumption at 10 dBm output threatened flight time. Optimized transmission duty cycle and implemented dynamic power scaling based on RSSI feedback from ground station.

% More images
% With captions:
\ImageRowTwoCaptioned{radio_latency.png}{radio_completed_transfer.png}{Radio latency measurements}{UART logs showing transfer connection}

% Or without captions:
% \ImageRowTwo{rf_performance_graph.jpg}{packet_structure_diagram.jpg}

\subsection*{Testing \& Validation}

Conducted comprehensive field testing across multiple environments:
\begin{itemize}
  \item \textbf{Open Field Tests:} Validated 3.2 km maximum range with line-of-sight
  \item \textbf{Urban Environment:} Achieved 1.1 km range with building obstructions
  \item \textbf{Interference Testing:} Verified operation in 2.4 GHz WiFi saturated areas
  \item \textbf{Reliability Metrics:} 99.2\% packet delivery, <80ms average latency for alerts
\end{itemize}

% With caption:
\SingleImageCaptioned{distance_test.png}{Field testing demonstrating 3+ km operational range. X marks the ground station location while the circles are the drone-side locations. }

% Or without caption:
% \SingleImage{drone_field_testing.jpg}

\subsection*{Key Takeaways \& Future Work}

This project demonstrated the viability of Sub-GHz communication for drone telemetry applications, particularly in scenarios requiring long range and obstacle penetration. Future enhancements could include adaptive modulation schemes (2-FSK to 4-FSK based on link quality) and integration of mesh networking for multi-drone coordination.

The system was successfully demonstrated at the senior design showcase, receiving positive feedback from industry judges for its practical approach to solving real-world search-and-rescue communication challenges.

% ============================================================
% FEATURED PROJECT 2: NPM CLONE (1-2 PAGES)
% ============================================================
\newpage
\begin{FeaturedProject}
  {NPM Package Registry Clone — Private Enterprise Software Distribution}
  {ECE 461: Software Engineering — Full-Stack \& Cloud Architecture Lead}
  {TypeScript, Node.js, AWS Lambda, AWS S3, DynamoDB, API Gateway, GitHub Actions, Jest, Winston}
  {Build a scalable, secure, private package management system for enterprise use that evaluates and stores software packages with automated quality metrics, access controls, and a web-based interface—essentially creating a private alternative to the public npm registry.}
  {Designed and implemented a serverless architecture on AWS with 8 RESTful API endpoints supporting full CRUD operations for package management. Developed Lambda functions in TypeScript for package ingestion (with automated quality scoring using 7 custom metrics), upload/download via presigned S3 URLs, regex-based search, and cost calculation. Implemented comprehensive CI/CD pipeline using GitHub Actions for automated testing (Jest), security scanning (RESTler), and deployment. Built responsive web frontend with S3 static hosting and integrated ADA-compliant UI (WCAG 2.1 AA).}
  {%
    \gitem 8 REST endpoints with complete package lifecycle management
    \gitem Automated quality scoring across 7 metrics (ramp-up, responsiveness, bus factor, etc.)
    \gitem Serverless architecture scaling within AWS free tier
    \gitem Presigned-URL security for time-limited S3 access
  }
  
  \vspace{6pt}
  \LinkTag{https://github.com/AviatorNic28/ECE-461-Phase-2}{GitHub Repository}
  \LinkTag{https://drive.google.com/file/d/1-CU-heo0OBlgMCqzhl4IIAyjx4r8V4bZ/view?usp=sharing}{Final Report}
\end{FeaturedProject}

\subsection*{System Architecture}

The system was architected as a fully serverless application leveraging AWS managed services to minimize operational overhead while maintaining scalability and security.

\textbf{Core Components:}
\begin{itemize}
  \item \textbf{API Gateway:} RESTful endpoint routing with CORS configuration
  \item \textbf{AWS Lambda:} 12+ serverless functions handling business logic (TypeScript/Node.js)
  \item \textbf{S3 Buckets:} Package storage with presigned URL access control
  \item \textbf{DynamoDB:} Two tables for package metadata and quality metrics
  \item \textbf{CloudWatch:} Centralized logging and monitoring
\end{itemize}

% Architecture diagram
% With caption:
\SingleImageCaptioned{AWS_structure.png}{Serverless AWS architecture with Lambda, S3, DynamoDB, and API Gateway}

% Or without caption:
% \SingleImage{npm_architecture_diagram.jpg}

\subsection*{Key Features \& Implementation Details}

\textbf{1. Package Upload with Security}

Implemented a two-stage upload process using presigned S3 URLs:
\begin{enumerate}
  \item Client requests upload → Lambda generates 10-minute presigned URL
  \item Client uploads directly to S3 → S3 trigger invokes metadata processor
  \item Metadata processor updates DynamoDB with package info
\end{enumerate}

This approach avoided Lambda payload size limits (6 MB) and improved security by limiting S3 access windows.

\textbf{2. Automated Package Quality Scoring}

Developed a rating system evaluating packages across 7 dimensions:
\begin{itemize}
  \item Ramp-up time (README quality, documentation)
  \item Responsiveness (issue/PR response times via GitHub API)
  \item Correctness (test coverage, build status)
  \item Bus factor (contributor distribution)
  \item License compliance (SPDX validation)
  \item Dependency pinning (security best practices)
  \item Code review fraction (PR approval rates)
\end{itemize}

Packages below 0.5 average score were automatically rejected during ingestion.

\textbf{3. Search \& Discovery}

Implemented regex-based search across package names and descriptions using DynamoDB Query with scan fallback for complex patterns.

% Screenshots
% With captions:
\SingleImageCaptioned{npm_frontend.png}{Website frontend}

% Or without captions:
% \ImageRowTwo{npm_frontend_ui.jpg}{npm_metrics_display.jpg}

\subsection*{Development Process \& Challenges}

\textbf{CI/CD Pipeline:}

Established GitHub Actions workflow automating:
\begin{itemize}
  \item Jest unit tests (targeting 80\%+ coverage)
  \item TypeScript compilation and linting
  \item Automated deployment to AWS (Lambda, S3, DynamoDB)
  \item RESTler security testing against OpenAPI spec
\end{itemize}

\textbf{Major Technical Challenge: Integration Issues}

Mid-project, switched from AWS Amplify to manual GitHub Actions deployment due to HTTPS/HTTP conflicts breaking frontend-backend communication. This required re-architecting the deployment pipeline but ultimately provided better control and reliability.

\textbf{Team Collaboration:}

Worked in a 4-person team using Agile methodology:
\begin{itemize}
  \item Weekly sprints with Discord standups
  \item Git feature branching with PR reviews
  \item Shared AWS account with IAM role separation
  \item Documentation in GitHub Wiki
\end{itemize}

\subsection*{Security Analysis (STRIDE Model)}

Conducted comprehensive threat modeling:
\begin{itemize}
  \item \textbf{Spoofing:} Mitigated with API key authentication (planned X-Authorization tokens)
  \item \textbf{Tampering:} HTTPS for all communication, presigned URLs for S3
  \item \textbf{Repudiation:} CloudWatch logging for audit trails
  \item \textbf{Information Disclosure:} IAM policies restricting DynamoDB access
  \item \textbf{DoS:} API Gateway rate limiting (not fully implemented)
  \item \textbf{Elevation of Privilege:} AWS IAM with principle of least privilege
\end{itemize}

% Testing results
% With captions:
\SingleImageCaptioned{cicd_deployment.png}{GitHub Actions CI/CD pipeline execution}

\CaptionedImageLarge{Accesibility_check.png}{WGAC Accessibility Test}

% Or without captions:
% \ImageRowTwo{npm_restler_results.jpg}{npm_cicd_pipeline.jpg}

\subsection*{Outcomes \& Lessons Learned}

Successfully delivered a functional package registry demonstrating:
\begin{itemize}
  \item Serverless architecture design and implementation
  \item RESTful API development with proper HTTP semantics
  \item Cloud infrastructure management (IaC principles)
  \item Security-first design with threat modeling
  \item Team collaboration in a complex software engineering project
\end{itemize}

\textbf{Key Lesson:} Integration testing should be prioritized earlier in development. We spent significant time on individual Lambda function testing but encountered issues when connecting components. Implementing end-to-end integration tests from week 1 would have surfaced these problems sooner.

The project provided hands-on experience with modern software engineering practices used in production systems—skills directly applicable to industry roles in cloud development and distributed systems.

% ============================================================
% ADDITIONAL PROJECTS (COMPACT CARDS)
% ============================================================
\section*{Additional Projects}

\begin{ProjectCardMultiImage}
  {Data-Regularized Q-Learning for Snake}
  {ECE 570 — Reinforcement Learning Research}
  {Python, PyTorch, OpenAI Gym, Reinforcement Learning, Computer Vision}
  {Evaluate whether data augmentation techniques (DrQ) improve training efficiency and stability in discrete-action, pixel-based RL environments.}
  {Implemented DrQ framework with random shift augmentation, double Q-learning, and custom Snake environment (84×84 pixel observations). Trained convolutional Q-networks with augmented replay buffers. Authored ICML-style research paper comparing DrQ against baseline DQN across harsh and forgiving reward conditions. Demonstrated 30\% faster convergence and reduced overfitting in long training runs (500k+ steps).}
  {\LinkTag{https://github.com/Duhuhruv/ECE570-DrQSnake}{GitHub}
  \LinkTag{https://drive.google.com/file/d/1VV7r5jJUt6nuBxgADgMndOKDJUNj06Iz/view?usp=sharing}{Paper}
  \LinkTag{https://www.youtube.com/watch?v=Rc4yFxNeazg}{Video}}
  
  \ImageRowTwoCaptioned{DrQRSFull1.png}{DQNRSFull1.png}{DrQ in a forgiving environment}{DQN (Baseline) in a forgiving environment}
  
\end{ProjectCardMultiImage}

\begin{ProjectCardMultiImage}
  {EPICS: Rovers for Aero \& Astro Education Team / Hippotherapy for Assistive Technology Team}
  {Purdue EPICS — Project Liaison \& Embedded Systems Lead}
  {C/C++, Arduino/STM32, Servo Control, Sensor Integration, Community Partnership}
  {Develop accessible, field-tested assistive robotics for hippotherapy (equine-assisted therapy) serving children with special needs through a year-long community partnership.
  
  Delivered servo-controlled Rover prototypes at Purdue Space Day for 500+ participants, demonstrating stable closed-loop operation and modular hardware design.}
  {Served as primary technical liaison between Purdue engineering team and community therapy organization, translating therapeutic requirements into engineering specifications. Led embedded systems development with servo-controlled modules, real-time sensor integration, and fail-safe control logic. Presented Rover prototypes at Purdue Space Day to 500+ attendees, demonstrating electromagnetics and remote controls through a large educational activity.}
  {\LinkTag{https://engineering.purdue.edu/EPICS}{EPICS Program}}
  
  \ImageRowTwoCaptioned{rover.png}{space_day.png}{Rover and Controller}{Rover demonstration at Purdue Space Day with 500+ attendee}
\end{ProjectCardMultiImage}

% ============================================================
% PROFESSIONAL EXPERIENCE
% ============================================================
\newpage
\section*{Professional Experience}

\textbf{Aerospace \& Defense Intern I/II} \hfill \textit{SeaTec Consulting, Bellevue, WA}\\
\textit{May 2022 – Aug 2023}
\begin{itemize}
  \item Automated a Boeing IP documentation program by developing a Python and VBA pipeline to process 18K+ technical files, cutting the contract-allocated 6-hour review time per document to under 1 minute and saving over \textbf{10,000 labor hours} across the scope.
  \item Created automation scripts in Access and Excel to streamline configuration management workflows and error tracking for aircraft system documentation.
  \item Streamlined airspace data-sharing documentation for NASA’s Urban Air Mobility Challenge by consolidating cross-team research and verification workflows.
\end{itemize}

% ============================================================
% SKILLS
% ============================================================
\section*{Technical Skills}

\begin{multicols}{2}
\SkillCategory{Languages}{C, C++, Python, TypeScript, Java, MATLAB, RISC-V Assembly}

\SkillCategory{Embedded \& Hardware}{TI CC1312R (SimpleLink), STM32, Arduino, UART/SPI/I2C, FreeRTOS}

\SkillCategory{RF \& Wireless}{Sub-GHz Protocols, 2-FSK/4-FSK Modulation, Link Budget Analysis}

\SkillCategory{Software Engineering}{REST APIs, Serverless Architecture, Distributed Systems, CI/CD}

\SkillCategory{Cloud \& DevOps}{AWS (Lambda, S3, DynamoDB, EC2, API Gateway), GitHub Actions}

\SkillCategory{AI/ML}{PyTorch, Reinforcement Learning, CNNs, Data Augmentation, Model Training}

\SkillCategory{Testing \& Tools}{Jest, RESTler, Git, Linux, GDB, Valgrind, Winston Logging}

\SkillCategory{Development Methods}{Agile/Scrum, Code Review, Technical Documentation, TDD}
\end{multicols}

% ============================================================
% EDUCATION
% ============================================================
\section*{Education}

\textbf{Purdue University} \hfill \textit{West Lafayette, IN}\\
Bachelor of Science in Computer Engineering \hfill \textit{Aug 2021 – May 2025}

\textbf{Relevant Coursework:}
Data Structures \& Algorithms, Operating Systems, Computer Networks, Embedded Systems Design, AI/Machine Learning, Digital Systems Design, Object-Oriented Programming (C++), Python for Data Science, Signals \& Systems, Probabilistic Methods

\textbf{Honors \& Activities:}
\begin{itemize}
  \item Dean's List (Dec 2021)
  \item Perfect Volunteer Attendance – Outdoors for All / Special Olympics (2020)
  \item Engineering Projects in Community Service (EPICS) — 2 semesters
\end{itemize}

% ============================================================
% FOOTER
% ============================================================
\vspace{12pt}
\divider
\begin{center}
  {\small\color{secondarygray}
    This portfolio showcases selected projects demonstrating capabilities across software engineering,\\
    embedded systems, RF communication, cloud infrastructure, and AI/ML.\\[3pt]
    Full project archive and source code: \href{https://github.com/Duhuhruv}{\color{linkblue}\textbf{github.com/Duhuhruv}}\\[3pt]
    \textit{Portfolio compiled: \today}
  }
\end{center}

\end{document}
